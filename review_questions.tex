\documentclass[11pt]{article}
\usepackage[margin=1in]{geometry}
\usepackage{setspace}
\usepackage{enumitem}

% Professional font setup
\usepackage[T1]{fontenc}
\usepackage{mathptmx}  % Times New Roman for text
\usepackage[scaled=0.9]{helvet}  % Helvetica for sans-serif
\usepackage{courier}   % Courier for monospace
\renewcommand{\familydefault}{\rmdefault}  % Use Times as default

\title{Review Questions and Solutions\\Dynamic IoT Applications and Isomorphic IoT Systems Using WebAssembly}
\author{Uddhav P. Gautam, Ram Mude\\
\small upgautam@vt.edu, ramm@vt.edu}
\date{}

\begin{document}
\maketitle

\singlespacing

\section*{Question 1: Dynamic Updates vs. Firmware Updates}

\textbf{Question:} How does the proposed method for "Dynamic IoT Applications" differ from traditional over-the-air (OTA) firmware updates in terms of mechanism and downtime?

\textbf{Solution:}

Traditional firmware updates typically require replacing the entire system image and rebooting the operating system, which causes significant downtime (measured at over 10 seconds in the study). In contrast, the proposed method splits the application into a core component and a WebAssembly (Wasm) runtime. Updates are applied by simply replacing the Wasm binary and reloading the runtime, which allows the device to update its behavior in approximately 1.4 seconds without a full system reboot.

\section*{Question 2: Isomorphism}

\textbf{Question:} What is meant by an “isomorphic IoT system,” and why is it beneficial?

\textbf{Solution:}
An isomorphic IoT system is one where the same Wasm binary, built from a single codebase, runs on the IoT device, edge servers, and cloud platforms. This is beneficial because it simplifies development and maintenance, ensures consistent behavior across layers, and reduces the need to rewrite logic for different architectures. It also enables easier deployment of machine learning inference or other logic uniformly across the whole IoT stack.

\end{document}
